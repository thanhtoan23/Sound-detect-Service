\section{Giới thiệu}

Phân loại âm thanh môi trường là một bài toán quan trọng trong IoT và các hệ thống thông minh hiện đại. Mục tiêu của đồ án là xây dựng một dịch vụ phân loại âm thanh (\textit{Sound Detection Service}) hoạt động trong thời gian thực, kết hợp xử lý tín hiệu số (DSP) và trí tuệ nhân tạo (AI) để nhận diện các loại âm thanh môi trường từ dữ liệu thu thập bằng thiết bị ReSpeaker Mic Array v2.0.

Báo cáo được tổ chức theo các nội dung chính sau:
\begin{itemize}
    \item Cơ sở lý thuyết: Trình bày lý thuyết DSP (bộ lọc Butterworth, AGC) và các phương pháp huấn luyện mô hình CNN cho phân loại âm thanh.
    \item Thiết kế hệ thống: Mô tả kiến trúc pipeline với bốn thành phần chính - thu thập âm thanh (VAD, DOA), xử lý tín hiệu (bandpass filter, spectral gating, AGC), phân loại AI (CNN), và giao diện (GUI/CLI).
    \item Triển khai và kiểm thử: Chi tiết quá trình thực thi các module, cấu hình tham số, và đánh giá kết quả trên dữ liệu test.
    \item Kết luận và hướng phát triển: Tóm tắt kết quả đạt được, những hạn chế hiện tại, và các đề xuất cải thiện trong tương lai.
\end{itemize}
