\section{Kết luận}

\subsection{Tóm tắt kết quả đạt được}

Đồ án đã xây dựng thành công một hệ thống phân loại âm thanh môi trường hoạt động trong thời gian thực với kiến trúc pipeline tích hợp từ phần cứng đến phần mềm. Hệ thống bao gồm các module thu thập âm thanh từ ReSpeaker Mic Array v2.0, xử lý tín hiệu số, phân loại bằng mô hình học sâu và giao diện tương tác cho người dùng.

Pipeline xử lý tín hiệu kết hợp các kỹ thuật DSP truyền thống nhằm làm sạch và chuẩn hóa tín hiệu đầu vào, trong khi mô hình CNN được huấn luyện trên đặc trưng log-mel spectrogram để học các đặc trưng âm thanh có ý nghĩa. Thông qua việc áp dụng các kỹ thuật regularization và data augmentation, mô hình đạt độ chính xác tốt trên tập kiểm tra và duy trì hiệu suất ổn định khi làm việc với dữ liệu có nhiễu.

Hệ thống cuối cùng có khả năng nhận diện 11 lớp âm thanh môi trường và đáp ứng yêu cầu vận hành trong thời gian thực.

\subsection{Đánh giá hiệu suất và ưu điểm}

Hệ thống thể hiện một số ưu điểm nổi bật:

\begin{itemize}
    \item Độ trễ thấp và khả năng hoạt động thời gian thực: Việc xử lý theo từng khung âm thanh cố định giúp hệ thống phản hồi nhanh, phù hợp cho các ứng dụng tương tác.
    \item Thiết kế mô-đun: Các thành phần của hệ thống được tách biệt rõ ràng, giúp việc bảo trì, mở rộng hoặc thay thế từng module trở nên thuận tiện.
    \item Cách tiếp cận hybrid hiệu quả: Sự kết hợp giữa DSP và học sâu giúp giảm nhiễu đầu vào, cải thiện tính ổn định của đặc trưng và nâng cao khả năng tổng quát hóa của mô hình.
\end{itemize}

\subsection{Những hạn chế}

Bên cạnh các kết quả đạt được, hệ thống vẫn tồn tại một số hạn chế:

\begin{itemize}
    \item Mô hình phụ thuộc vào chất lượng và phạm vi của tập dữ liệu huấn luyện, do đó hiệu suất có thể suy giảm khi gặp các âm thanh ngoài phân bố dữ liệu đã học.
    \item Hệ thống hiện được thiết kế cho bài toán phân loại đơn nhãn, nên chưa xử lý hiệu quả các tình huống có nhiều nguồn âm thanh đồng thời.
    \item Việc triển khai trên các thiết bị nhúng có tài nguyên hạn chế vẫn còn gặp khó khăn do yêu cầu về bộ nhớ và năng lực tính toán của mô hình.
    \item Các tham số xử lý và ngưỡng quyết định được thiết lập cố định, chưa có cơ chế tự thích ứng theo sự thay đổi của môi trường theo thời gian.
\end{itemize}

\subsection{Hướng phát triển tương lai}

Trong tương lai, hệ thống có thể được cải thiện theo các hướng sau:

\begin{itemize}
    \item Mở rộng và đa dạng hóa tập dữ liệu huấn luyện để nâng cao khả năng tổng quát hóa trong môi trường thực tế.
    \item Nghiên cứu các kiến trúc mô hình nâng cao như attention hoặc transformer nhằm khai thác tốt hơn mối quan hệ theo thời gian của tín hiệu âm thanh.
    \item Bổ sung cơ chế phát hiện âm thanh bất thường để tăng độ tin cậy khi vận hành ngoài thực tế.
    \item Áp dụng các kỹ thuật tối ưu hóa mô hình như quantization hoặc pruning để phục vụ triển khai trên thiết bị edge.
    \item Mở rộng bài toán sang phân loại đa nhãn nhằm xử lý các tình huống âm thanh chồng lấp.
\end{itemize}

\subsection{Kết luận cuối cùng}

Đồ án đã chứng minh tính khả thi của việc xây dựng một hệ thống phân loại âm thanh môi trường thời gian thực theo cách tiếp cận kết hợp giữa xử lý tín hiệu số và học sâu. Mặc dù còn tồn tại một số hạn chế, hệ thống hiện tại cung cấp một nền tảng vững chắc cho các nghiên cứu và phát triển tiếp theo, hướng tới các ứng dụng thực tế như nhà thông minh, giám sát an ninh và hỗ trợ con người trong môi trường sống.

\subsection*{6.6.    Dataset}
\href{https://github.com/karolpiczak/ESC-50}{Đây là đường dẫn đến tập dữ liệu được nhóm sử dụng trong quá trình huấn luyện mô hình}